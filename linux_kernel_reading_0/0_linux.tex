\documentclass[xcolor=svgnames]{beamer}
\mode<presentation>
{
      \setbeamertemplate{footline}[page number]
      \setbeamercovered{transparent}
      \setbeamertemplate{navigation symbols}{}
      \usecolortheme[named=DarkGreen]{structure}
}

\usepackage[english]{babel}
\usepackage{times}
\usepackage{url}
\usepackage{CJKutf8}
\usepackage{graphics}
\usepackage{amsmath}
\usepackage{listings}
\lstset{breakatwhitespace,
language=C++,
columns=fullflexible,
keepspaces,
breaklines,
tabsize=3,
showstringspaces=false,
extendedchars=true}


\begin{document}
\begin{CJK*}{UTF8}{gbsn}


\title{Linux内核代码:简介}
\author{李中国}
\institute{苏州大学计算机科学与技术学院} %\\ 18862301730 \\ 理工楼409}

\begin{frame}
\maketitle
\end{frame}

\begin{frame}[fragile]
\frametitle{Linux内核源代码目录树}
\begin{tabular}{|r l|} 
\hline 
目录 & 描述 \\
\hline 
arch & 硬件体系结构相关的代码 \\
Documentation & 内核源代码文档 \\
drivers & 设备驱动程序 \\
fs & 文件系统 \\
include & 内核头文件 \\
ipc & 进程间通讯代码\\
init & 系统启动及初始化代码 \\
kernel & 核心代码(如进程调度器等)\\
mm & 内存管理系统、虚拟内存系统 \\
net & 网络系统代码(如TCP/IP协议栈)\\
lib & 辅助函数库 \\
\hline 
\end{tabular} 
\end{frame}

\begin{frame}[fragile]
\frametitle{Linux内核与普通应用程序的区别}
\begin{itemize}
\item 内核不能使用C语言标注库函数(及C标准头文件)
\item Linux内核编程使用的是GNU C而不是标准C语言
\item 内核代码缺乏内存保护机制(应用程序则有)
%\item 内核无法方便地使用浮点数操作指令
\item 与应用程序比,内核运行时的栈非常小
\item 内核代码必须正确处理同步与并发问题
\item 可移植性对内核而言非常重要
\end{itemize}
\end{frame}

\begin{frame}[fragile]
\frametitle{为什么内核不能使用C语言库函数及头文件?}
\begin{block}{原因}
\begin{itemize}
\item chicken-and-egg问题
\item C标准库对内核而言太庞大、低效
\end{itemize}
\end{block}
\begin{block}{Linux内核的应对策略}
\begin{itemize}
\item 自己实现部分库函数功能:\verb|lib/string.c|,对应头文件\verb|<linux/string.h>|
\item 标准库中printf函数在内核中对应的是printk
\begin{itemize}
\item 例如 \verb|printk(KERN_ERR "this is an error!\n")|
\end{itemize}
\end{itemize}
\end{block}
\end{frame}

\begin{frame}[fragile]
\frametitle{Linux内核使用GNU C编程}
\begin{itemize}
\item GNU C支持内联函数(与宏相比有什么有点?)
\begin{itemize}
\item 内联函数在使用前需先定义;一般置于头文件中
\item 例:\verb|static inline void wolf(unsigned long t)|
\end{itemize}
\item 内联汇编语言: 可以在C程序中直接使用汇编语句
\begin{itemize}
\item 例:\verb|asm volatile("rdtsc":"=a"(l), "=d"(h));|
\end{itemize}
\item GNU C支持条件分支优化,例如:
\begin{verbatim}
/* 'error' is nearly always zero */
if (unlikely(error)) {
    ... 
}

/* 'success' is nearly always nonzero */
if (likely(success)) {
    ... 
}

\end{verbatim}
\end{itemize}
\end{frame}

\begin{frame}[fragile]
\frametitle{内核中没有内存保护机制}
\begin{itemize}
\item 应用程序访问非法内存地址?
\begin{itemize}
\item 操作系统会给该进程发送\verb|SIGSEGV|信号并将其杀死
\end{itemize}
\item 谁来阻止操作系统访问非法地址?
\item 内核所占的物理内存无法进行分页
\begin{itemize}
\item 所以内核在使用物理内存时必须特别节俭
\end{itemize}
\end{itemize}
\end{frame}

\begin{frame}[fragile]
\frametitle{核心态下的栈空间}
\begin{itemize}
\item 应用程序可以使用的栈是动态增长的、具体大小仅由地址空间限制
\item 内核可以利用的栈大小是固定的(2 pages):
\begin{itemize}
\item 32位x86平台上,固定为8K
\item 64位平台上固定为16K
\end{itemize}
\item 内核在编译前可以配置其所利用的栈为4K
\end{itemize}
\end{frame}
\end{CJK*}
\end{document}
