\documentclass[xcolor=svgnames]{beamer}
\mode<presentation>
{
      \setbeamertemplate{footline}[page number]
      \setbeamercovered{transparent}
      \setbeamertemplate{navigation symbols}{}
      \usecolortheme[named=DarkGreen]{structure}
}

\usepackage[english]{babel}
\usepackage{times}
\usepackage{url}
\usepackage{CJKutf8}
\usepackage{graphics}
\usepackage{amsmath}

\begin{document}
\begin{CJK*}{UTF8}{gbsn}


\title{内存管理}

%\begin{frame}
%\maketitle
%\end{frame}

\begin{frame}{第6次编程作业:寻找互素数对}
%\maketitle
\begin{equation}
4 \perp 9, \quad 34 \perp 35
\end{equation}

\begin{equation}
\Pr(m \perp n) = \frac{6}{\pi^2} \approx 0.607927102
\end{equation}

\end{frame}

\begin{frame}{第6次编程作业:寻找互素数对}
%在Intel Core2 Duo上,计算100000以内(共100亿个数对)互素的数对个数,
%结果是6079301507(理论预测值为6079271020,误差万分之0.05), 用时18分钟。
\begin{itemize}
\item 硬件平台:Intel Core2 (Duo)
\item 计算100000以内互素的数对的个数
\item 共涉及100亿个数对
\item 计算结果:6079301507
\item 理论预测:6079271020
\item 相对误差:万分之0.05
\item 耗时:18分钟
\item 本次作业:用多线程编程技术加速
\end{itemize}
\end{frame}

\begin{frame}{Parkinson's law}
\begin{columns}%[t]
\column{.5\textwidth}
\begin{enumerate}
\item Work expands so as to fill the time available for its completion.
\item[]
\item Data expands to fill the space available for storage.
\end{enumerate}
\column{.5\textwidth}
\includegraphics[width=1.0\textwidth]{parkinson.jpg}
\end{columns}%[t]
\end{frame}

\begin{frame}{程序员希望的内存}
\begin{columns}%[t]
\column{.5\textwidth}
\begin{block}{联系上次作业}
私有的、无穷大、无穷快、便宜、持久性
\end{block}
\column{.5\textwidth}
\includegraphics[width=1.0\textwidth]{memory_hier.jpg}
\end{columns}%[t]
\end{frame}

\begin{frame}{内存管理器(Memory Manager)的任务}
\begin{itemize}
\item 提供内存抽象界面
\item 分配物理内存,回收物理内存
\item 记录内存使用情况等
\end{itemize}
\end{frame}

\begin{frame}{没有内存抽象:程序员直接操作物理内存}
\includegraphics[width=1.0\textwidth]{nomem.png}

a. 内存中一次只能驻留一个程序:\alert{MOV REGISTER1, 1000}

b. 操作系统自身代码难以保护(没有地址空间概念)
\end{frame}

\begin{frame}{虚拟内存技术}
%虚拟内存技术可解决个问题:
\begin{itemize}
\item 问题一:如何让多个程序驻留内存?
\begin{itemize}
\item 每个程序有专属地址空间
\item 程序不能非法访问其他程序的地址空间
\end{itemize}
\item 问题二:如何满足程序对内存的无限需求?
\begin{itemize}
\item 地址空间分成若干\alert{页面}
\item 地址空间的页面映射到物理内存的\alert{页框}内
\item 利用\alert{页表}实现页面号到页框号的映射
\item 不是所有的页面都需要放到物理内存中
\item 缺页中断技术
\end{itemize}
\end{itemize}
\end{frame}

\begin{frame}{虚拟地址、物理地址及内存管理单元}
\includegraphics[width=0.9\textwidth]{mmu.png}
\end{frame}

%\begin{frame}{系统调用的例子}
%\includegraphics[width=0.9\textwidth]{examples.jpg}
%\end{frame}

\begin{frame}{虚拟地址、物理地址及内存管理单元}
\begin{columns}%[t]
\column{.4\textwidth}
\begin{itemize}
\item 虚拟地址空间:64K (16 bit)
\item 物理地址空间:32K (15 bit)
\item 页面大小:4K 
\item 共16个(虚拟)页面,8个(物理)页框
\item 对于大于32K的程序,只能有32K驻留物理内存(右图数字部分)
\end{itemize}
\column{.6\textwidth}
\includegraphics[width=1.0\textwidth]{vm.png}
\end{columns}%[t]
\end{frame}

\begin{frame}{虚拟地址、物理地址及内存管理单元}
\begin{columns}%[t]
\column{.4\textwidth}
\begin{block}{MOV REG, 20500}
20500 = 20K + 20 

页面5 $\rightarrow$ 页框3

12K + 20 = 12308(物理地址)
\end{block}
\begin{block}{MOV REG, 32780}
对应虚拟页面8(缺页)

\alert{what next?}
\end{block}
\column{.6\textwidth}
\includegraphics[width=1.0\textwidth]{vm.png}
\end{columns}%[t]
\end{frame}

\begin{frame}{虚拟地址、物理地址及内存管理单元}
\begin{columns}%[t]
\column{.4\textwidth}
虚拟地址映射到物理地址,考虑右图例子:
\begin{itemize}
\item 页面大小:4K 
\item 页内地址为12位
\item 对于16位机器而言,有4位用于页表索引
\item 因此共有16个虚拟页面
\item 8个物理页框(需3位)
\end{itemize}
\column{.6\textwidth}
\includegraphics[width=1.0\textwidth]{mmu2.png}
\end{columns}%[t]
\end{frame}

\begin{frame}{页表项}
\includegraphics[width=1.0\textwidth]{pte.png}
\begin{description}
\item[page frame number]描述该页表项对应的页框编号
\item[present / absent]表示该页表是否在内存中
\item[protection]含读、写、执行等权限信息
\item[modified]该页表内容是否被修改过
\item[referenced]该页表内容是否被用过(读写)
\item[caching disabled]用于memory mapped I/O(后续)
\end{description}
\end{frame}

\begin{frame}{虚拟内存技术与分页技术面临的两大问题}
\begin{enumerate}
\item 从虚拟地址到物理地址的映射必须快
\begin{itemize}
\item 每条指令都需从内存取出
\item 大量指令涉及读写内存(CISC机器)
\end{itemize}
\item 如果虚拟地址空间很大,则页表规模会特别大

考虑页面大小4K的虚拟内存系统:
\begin{itemize}
\item 32位虚拟地址: 100万个页表项
\item 64位虚拟地址:??
\end{itemize}
\end{enumerate}
\end{frame}

\begin{frame}{从虚拟地址到物理地址的快速映射:TLB(联想式存储)}
\includegraphics[width=1.0\textwidth]{tlb.png}

放置在MMU里面。来虚拟地址,先查TLB。
\begin{itemize}
\item 若能在TLB中查到该虚拟地址,则直接输出物理页框号码
\item 否则,去内存中的页表中查找对应页框号码,并将其调入TLB
\end{itemize}
\end{frame}

\begin{frame}{处理大规模地址空间的方法一:多级页表}
\includegraphics[width=0.7\textwidth]{multi.png}
\end{frame}

\begin{frame}{处理大规模地址空间的方法二:倒排页表}
\includegraphics[width=1.0\textwidth]{inverted.png}
\end{frame}

\begin{frame}{页面置换算法}
当发生缺页时,OS需要选择一个页面将其从物理内存转移至硬盘,然后从硬盘调入所缺页面。
\end{frame}

\begin{frame}{老化(aging)页面置换算法}
\includegraphics[width=1.0\textwidth]{aging.png}
\end{frame}

\begin{frame}{工作集(working set)页面置换算法}
\begin{itemize}
\item 访问的\alert{局部性}:在任一时间段内,程序仅仅访问其所有页面的一小部分。
\item[]
\item 我们将程序在某时间段内密集访问的页面集合成为\alert{工作集}
\end{itemize}
\end{frame}

\begin{frame}{工作集(working set)页面置换算法}
\includegraphics[width=1.0\textwidth]{ws0.png}
\end{frame}

\begin{frame}{工作集(working set)页面置换算法}
\includegraphics[width=1.0\textwidth]{ws.png}
\end{frame}

\begin{frame}{WSClock页面置换算法}
\includegraphics[width=1.0\textwidth]{wsclock1.png}
\end{frame}

\begin{frame}{WSClock页面置换算法}
\includegraphics[width=1.0\textwidth]{wsclock2.png}
\end{frame}
%\begin{frame}{对临界资源的互斥访问}
%\includegraphics[width=1.0\textwidth]{mutual.png}
%\end{frame}

%\begin{frame}{信号量机制(Semaphores)}
%为了解决唤醒信号丢失的问题,引入信号量,它是一种特殊的整型变量。在信号量上定义两个\alert{原子操作}:
%\begin{description}
%\item[down]  如果信号量值大于0,则将其减1然后返回;否则,进程在该信号量上进入睡眠
%\item[up]  如果有进程在该信号量上睡眠,则选择其中一个唤醒;否则,信号量加1 
%\end{description}
%\end{frame}
%
%\begin{frame}{用信号量解决生产者--消费者问题}
%\begin{columns}[b]
%\column{.5\textwidth}
%\includegraphics[width=1.0\textwidth]{prodsem.png}
%\column{.5\textwidth}
%\includegraphics[width=1.0\textwidth]{conssem.png}
%\end{columns}%[t]

%该方案中,信号量empty和full具有计数和同步功能,而mutex仅有互斥功能。
%\end{frame}

%\begin{frame}{专门用来实现互斥的特殊信号量 -- 互斥锁}
%互斥锁只有两种状态:locked (1) / unlocked (0)
%
%\includegraphics[width=0.5\textwidth]{mutex.png}
%\end{frame}

%\begin{frame}{互斥锁与忙等待的区别}
%\begin{columns}[b]
%\column{.5\textwidth}
%\includegraphics[width=1.0\textwidth]{mutex.png}
%\column{.5\textwidth}
%\includegraphics[width=1.0\textwidth]{tsl.png}
%\end{columns}%[t]
%后者:不断利用CPU指令测试临界资源,直至时间片用光被从CPU上撤下来
%\end{frame}
%
%\begin{frame}{信号量的危险情形 --- 管程机制的引入}
%\begin{columns}[b]
%\column{.5\textwidth}
%\includegraphics[width=1.0\textwidth]{prodsem.png}
%\column{.5\textwidth}
%\includegraphics[width=1.0\textwidth]{conssem.png}
%\end{columns}%[t]
%\alert{危险:}如果程序员不小心把producer中的down(empty)和down(mutex)顺序颠倒,
%则当缓冲区满时,会发生什么?
%\end{frame}

%\begin{frame}{信号量的危险情形 --- 管程机制的引入}
%\begin{itemize}
%\item 发生死锁。
%\item[]
%\item 因此,最好由编译器自动处理这种容易出错的程序段。---引入管程。
%\item[]
%\item 对比:C++中构造函数与析构函数
%\end{itemize}
%\end{frame}

%\begin{frame}{管程:解决生产者--消费者问题}
%\begin{columns}[b]
%\column{.5\textwidth}
%\includegraphics[width=1.0\textwidth]{mon1.png}
%\column{.5\textwidth}
%\includegraphics[width=1.0\textwidth]{mon2.png}
%\end{columns}%[t]
%
%\alert{注意概念:} 条件变量empty, full以及wait, signal
%
%此外,insert与remove之间的互斥由编译器完成
%\end{frame}

%\begin{frame}{管程:解决生产者--消费者问题}
%\begin{itemize}
%\item 管程内程序段之间的互斥(自动)
%\item 进程同步问题?:条件变量及wait, signal实现
%\begin{itemize}
%\item wait: 将当前进程阻塞,并允许其他进程进入管程
%\item signal: 将被相应条件变量阻塞的进程唤醒
%%\end{itemize}
%\item 上述方法中,signal必须是最后一条指令,为什么? 
%\end{itemize}
%\end{frame}

%\begin{frame}{消息传递机制:解决不同机器上进程间同步问题}
%\begin{columns}[b]
%\column{.5\textwidth}
%\includegraphics[width=1.0\textwidth]{msgprod.png}
%\column{.5\textwidth}
%\includegraphics[width=1.0\textwidth]{msgcons.png}
%\end{columns}%[t]
%\end{frame}


\end{CJK*}
\end{document}
