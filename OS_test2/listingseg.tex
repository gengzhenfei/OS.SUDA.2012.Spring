% listingseg.tex
% Examples of the listings package for code listings,
% essentially from the package documentation file
% John Leis
% Feb 2002
% Updated Oct 2003
% Updated Oct 2006

\documentclass{article}

%---------------------------------------------------------------
\usepackage{a4page}
\usepackage{listings}
\usepackage{color}
%---------------------------------------------------------------

\begin{document}

%---------------------------------------------------------------
\title{Typesetting Program Code Using the \texttt{listings} Package}
\author{John Leis}
\date{\today}
\maketitle
%---------------------------------------------------------------

%---------------------------------------------------------------
\lstlistoflistings
%---------------------------------------------------------------

%---------------------------------------------------------------
\definecolor{listinggray}{gray}{0.9}
%---------------------------------------------------------------

Only simple examples are shown here; see the package documentation file
for full descriptions/examples.

%---------------------------------------------------------------
\section{Listing embedded in \LaTeX\/ source}
%---------------------------------------------------------------

%---------------------------------------------------------------
\lstset{language=c++}
\lstset{commentstyle=\textit}
\begin{lstlisting}[frame=trbl]{}
    for(i = 0; i < 10; i++)
    {
        // increment the pointer
        *p++ = i;
    }
\end{lstlisting}
%---------------------------------------------------------------

%---------------------------------------------------------------
\section{Inline Code}
%---------------------------------------------------------------



\lstset{language=c}
the call \lstinline!socket()! creates a socket.

Using bfseries it is \lstinline[basicstyle=\bfseries]!socket()! which
distinguishes it from the surrounding text.

Using ttfamily it is  \lstinline[basicstyle=\ttfamily]!socket()! which
gives a small but helpful distinction from the surrounding text.


%---------------------------------------------------------------
\section{Background Color, Line Numbers}
%---------------------------------------------------------------

%---------------------------------------------------------------
% registered trademark symbol
\newcommand{\reg}{$\mbox{}^{\textregistered}$\hspace{1ex}}
%---------------------------------------------------------------

%---------------------------------------------------------------
\lstset{language=matlab}
%\lstset{backgroundcolor=listinggray}
\lstset{backgroundcolor=\color{listinggray}}
\lstset{linewidth=90mm}
\lstset{frameround=tttt}
%\lstset{frameround=trbl}
%\lstset{labelstep=1}
\lstset{keywordstyle=\color{red}\bfseries\underbar}
\begin{center}
    \begin{minipage}{100mm}
        %\begin{lstlisting}[frame=trBL,indent=10mm,caption=My MATLAB Code,label=lst:matlab,gobble=4]{}
        \begin{lstlisting}[frame=trBL,caption=Some MATLAB\reg Code,label=lst:matlab,gobble=4]{}
            for k=1:10
                array(k, :) = ones(4,1);
            end
        \end{lstlisting}
    \end{minipage}
\end{center}
%---------------------------------------------------------------

See Listing~\ref{lst:matlab}.


%---------------------------------------------------------------
\section{Imported Source File}
%---------------------------------------------------------------

%Listing from source file \texttt{SimpleButtons.java} \\
Listing from source file \texttt{lowertri.m} \\

%---------------------------------------------------------------
\lstset{language=matlab}
%\lstset{backgroundcolor=listinggray,framerulecolor=blue}
%\lstset{backgroundcolor=listinggray,rulecolor=blue}
\lstset{backgroundcolor=\color{listinggray},rulecolor=\color{blue}}
\lstset{linewidth=\textwidth}
%\lstset{labelstep=10}
%\lstset{commentstyle=\textit, stringstyle=\upshape,stringspaces=false}
\lstset{commentstyle=\textit, stringstyle=\upshape,showspaces=false}
\lstset{frame=trBL,frameround=tttt}
\lstinputlisting[caption=Some MATLAB Code,label=lst:lowertri]{lowertri.m}
%---------------------------------------------------------------


\clearpage
%---------------------------------------------------------------
\section{Continued Listings}
%---------------------------------------------------------------

%---------------------------------------------------------------
\lstset{language=c++}
\lstset{commentstyle=\textit}
%\lstset{labelstep=1}
%\lstset{backgroundcolor=,framerulecolor=}
\lstset{backgroundcolor=,rulecolor=}
\begin{lstlisting}[frame=tb]{somecode}
    for(i = 0; i < 10; i++)
    {
        // increment the pointer
        *p++ = i;
    }
\end{lstlisting}
%---------------------------------------------------------------

and later...
%---------------------------------------------------------------
\begin{lstlisting}[frame=tb]{somecode}
    for( i = 0; i < 10; i++ )
    {
        for( j = 0; j < 10; j++ )
        {
        }
    }
\end{lstlisting}
%---------------------------------------------------------------

%---------------------------------------------------------------
\section{Embedded Comments}
%---------------------------------------------------------------


% compile warning on \clearpage below
%---------------------------------------------------------------
%\lstset{texcl=true}        % latex comment lines
%\lstset{labelstep=0}
\begin{lstlisting}[frame=tb,texcl]{}
    // \upshape this code is \copyright
    for( i = 0; i < 10; i++ )
    {
    }
    // force continuation on next page \clearpage

    for( i = 0; i < 10; i++ )
    {
        for( j = 0; j < 10; j++ )
        {
            // \upshape calculate $a_{ij}$
            a[i][j] = b[j][i];
        }
    }
\end{lstlisting}
%---------------------------------------------------------------


%---------------------------------------------------------------
\section{Literate Programming}
%---------------------------------------------------------------

%---------------------------------------------------------------
\definecolor{lbcolor}{rgb}{0.9,0.9,0.9}
\lstset{language=c++}
\lstset{commentstyle=\textit}
%\lstset{backgroundcolor=lbcolor,framerulecolor=}
\lstset{backgroundcolor=\color{lbcolor},rulecolor=}
\lstset{literate={<=}{{$\le$}}{2}}
%\color{yellow}

\lstset{literate={=}{{$\leftarrow$}}{1}{<=}{{$\le$}}{2}{&&}{{$\cap$}}{2}}
\begin{lstlisting}[frame=tb]{}
    a = b && 0x0f;
    for(i = 0; i < 10; i++)
    {
        // increment the pointer
        *p++ = i;
        if( *p <= 0xff )
        {
        }
    }
\end{lstlisting}
%---------------------------------------------------------------



\end{document}
%---------------------------------------------------------------
