\documentclass[xcolor=svgnames]{beamer}
\mode<presentation>
{
      \setbeamertemplate{footline}[page number]
      \setbeamercovered{transparent}
      \setbeamertemplate{navigation symbols}{}
      \usecolortheme[named=DarkGreen]{structure}
}

\usepackage[english]{babel}
\usepackage{times}
\usepackage{url}
\usepackage{listings}
\usepackage{color}
\usepackage{CJKutf8}
\usepackage{graphics}

\begin{document}
\begin{CJK*}{UTF8}{gbsn}


\title{第二次课堂测试 -- 限时20分钟}

\maketitle

\begin{frame}{第1题}
在Linux 3.3.2内核源代码flow\_keys.h(如下)中,第1、2行和最后一行所起的作用是什么?
\lstset{language=C, frame=trbl}
\lstinputlisting{flowkeys.h}

\end{frame}

\begin{frame}{第2题}
以下Linux代码中的do和while(0)可否删掉?说明理由
\lstset{language=C, frame=trbl}
\lstinputlisting{fpu.h}
\alert{注意}:该宏在Linux源代码中是这样使用的:
\lstset{language=C, frame=trbl}
\lstinputlisting{test.c}

\end{frame}

\begin{frame}{第3题}
请说明什么是信号量以及down和up操作。
\end{frame}

\end{CJK*}
\end{document}

