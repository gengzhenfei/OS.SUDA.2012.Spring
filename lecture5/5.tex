\documentclass[xcolor=svgnames]{beamer}
\mode<presentation>
{
      \setbeamertemplate{footline}[page number]
      \setbeamercovered{transparent}
      \setbeamertemplate{navigation symbols}{}
      \usecolortheme[named=DarkGreen]{structure}
}

\usepackage[english]{babel}
\usepackage{times}
\usepackage{url}
\usepackage{CJKutf8}
\usepackage{graphics}

\begin{document}
\begin{CJK*}{UTF8}{gbsn}


\title{进程间的同步、互斥、通信以及调度}

%\begin{frame}{安装Ubuntu 11.10操作系统: ISO文件}
%\includegraphics[width=1.5\textwidth]{ubuntu-iso.png}
%\end{frame}

%\begin{frame}{课堂测试 -- 限时20分钟以内}
%\begin{enumerate}
%\item 请说明什么是用户态与核心态。
%\item[]
%\item 进程运行态、阻塞态及就绪态这三个状态之间在理论上有6种可能的相互转换,但其中有两种不可能存在,请问是哪两种并说明原因。
%\item[]
%\item 我们在编程作业中用到的函数, printf属于C语言库函数,write属于操作系统提供的系统调用函数。在没有其它任何信息的情况下,请你
%判断getpid()(获取进程编号)是库函数还是系统调用?说明你的理由。
%\item[]
%\item 什么是进程上下文切换?如果让你实现上下文切换功能,你会采用下列编程语言中的哪一种:Java, C, C++, 汇编语言.
%说明你选择该编程语言的原因。
%\end{enumerate}
%\end{frame}

%\begin{frame}{系统调用的例子}
%\includegraphics[width=0.9\textwidth]{examples.jpg}
%\end{frame}

\begin{frame}{临界资源与临界区}
\begin{columns}%[t]
\column{.4\textwidth}
考虑网络订票软件:
\begin{itemize}
\item 进程A发现3号车10C座位空闲
\item 此时操作系统调度进程B运行
\item 进程B同样发现该位子的票尚未售出, 于是将该票买给旅客
\item 进程A重新运行后,再次将3-10C售出
\end{itemize}
\column{.6\textwidth}
\includegraphics[width=1.0\textwidth]{booking.jpg}
\end{columns}%[t]
\end{frame}

\begin{frame}{对临界资源的互斥访问}
理想的互斥方案需要满足4个条件:
\begin{enumerate}
\item[]
\item 两个进程不能同时进入临界区
\item[]
\item 不能依赖CPU数目或者运行速度
\item[]
\item 不在临界区的进程,不能妨碍其他进程进入临界区
\item[]
\item 任一进程需在有限时间内能够进入临界区
\end{enumerate}
\end{frame}

\begin{frame}{对临界资源的互斥访问}
\includegraphics[width=1.0\textwidth]{mutual.png}
\end{frame}

\begin{frame}{对临界资源的互斥访问: 方法1 -- 交替进入临界区}
\includegraphics[width=1.0\textwidth]{alter.png}

缺陷:违反条件3 (设想进程A循环体运行1秒,进程B运行100秒)

进程A与B必须锁步(交替)进入临界区
\end{frame}

\begin{frame}{对临界资源的互斥访问: 方法2 -- 忙等待}
\begin{columns}%[t]
\column{.5\textwidth}
\begin{itemize}
\item 需要硬件支持TSL指令
\item[]
\item 进程进入临界区前,调用enter\_region
\item[]
\item 离开临界区时,调用leave\_region
\item[]
\item 这是一个正确的解决方法,但是 ...
\end{itemize}
\column{.5\textwidth}
\includegraphics[width=1.0\textwidth]{tsl.png}
\end{columns}%[t]
\end{frame}

\begin{frame}{对临界资源的互斥访问: 方法2 -- 忙等待}
两个缺点:

\begin{itemize}
\item 缺点1:忙等待浪费了CPU时间
\item 缺点2:优先级反转问题
\begin{itemize}
\item 进程H优先级高于进程L, 二者同时需要某临界资源
\item 假设当进程L在临界区时,进程H可以运行
\item 结局: 进程L永远无法离开临界区,H永远忙等待
\end{itemize}
\end{itemize}

为了克服这些缺点,增加sleep和wakeup系统调用
\end{frame}

\begin{frame}{生产者--消费者问题}
\begin{columns}[b]
\column{.5\textwidth}
\includegraphics[width=1.0\textwidth]{prod.png}
\column{.5\textwidth}
\includegraphics[width=1.0\textwidth]{consum.png}
\end{columns}%[t]

缺陷: 测试count==0成功后,消费者进程调用sleep之前,调度生产者进程运行...
\end{frame}

\begin{frame}{信号量机制(Semaphores)}
为了解决唤醒信号丢失的问题,引入信号量,它是一种特殊的整型变量。在信号量上定义两个\alert{原子操作}:
\begin{description}
\item[down]  如果信号量值大于0,则将其减1然后返回;否则,进程在该信号量上进入睡眠
\item[up]  如果有进程在该信号量上睡眠,则选择其中一个唤醒;否则,信号量加1 
\end{description}
\end{frame}

\begin{frame}{用信号量解决生产者--消费者问题}
\begin{columns}[b]
\column{.5\textwidth}
\includegraphics[width=1.0\textwidth]{prodsem.png}
\column{.5\textwidth}
\includegraphics[width=1.0\textwidth]{conssem.png}
\end{columns}%[t]

该方案中,信号量empty和full具有计数和同步功能,而mutex仅有互斥功能。
\end{frame}

\begin{frame}{专门用来实现互斥的特殊信号量 -- 互斥锁}
互斥锁只有两种状态:locked (1) / unlocked (0)

\includegraphics[width=0.5\textwidth]{mutex.png}
\end{frame}

\begin{frame}{互斥锁与忙等待的区别}
\begin{columns}[b]
\column{.5\textwidth}
\includegraphics[width=1.0\textwidth]{mutex.png}
\column{.5\textwidth}
\includegraphics[width=1.0\textwidth]{tsl.png}
\end{columns}%[t]
后者:不断利用CPU指令测试临界资源,直至时间片用光被从CPU上撤下来
\end{frame}

\begin{frame}{信号量的危险情形 --- 管程机制的引入}
\begin{columns}[b]
\column{.5\textwidth}
\includegraphics[width=1.0\textwidth]{prodsem.png}
\column{.5\textwidth}
\includegraphics[width=1.0\textwidth]{conssem.png}
\end{columns}%[t]

\alert{危险:}如果程序员不小心把producer中的down(empty)和down(mutex)顺序颠倒,
则当缓冲区满时,会发生什么?
\end{frame}

\begin{frame}{信号量的危险情形 --- 管程机制的引入}
\begin{itemize}
\item 发生死锁。
\item[]
\item 因此,最好由编译器自动处理这种容易出错的程序段。---引入管程。
\item[]
\item 对比:C++中构造函数与析构函数
\end{itemize}
\end{frame}

\begin{frame}{管程:解决生产者--消费者问题}
\begin{columns}[b]
\column{.5\textwidth}
\includegraphics[width=1.0\textwidth]{mon1.png}
\column{.5\textwidth}
\includegraphics[width=1.0\textwidth]{mon2.png}
\end{columns}%[t]

\alert{注意概念:} 条件变量empty, full以及wait, signal

此外,insert与remove之间的互斥由编译器完成
\end{frame}

\begin{frame}{管程:解决生产者--消费者问题}
\begin{itemize}
\item 管程内程序段之间的互斥(自动)
\item 进程同步问题?:条件变量及wait, signal实现
\begin{itemize}
\item wait: 将当前进程阻塞,并允许其他进程进入管程
\item signal: 将被相应条件变量阻塞的进程唤醒
\end{itemize}
\item 上述方法中,signal必须是最后一条指令,为什么? 
\end{itemize}
\end{frame}

\begin{frame}{消息传递机制:解决不同机器上进程间同步问题}
\begin{columns}[b]
\column{.5\textwidth}
\includegraphics[width=1.0\textwidth]{msgprod.png}
\column{.5\textwidth}
\includegraphics[width=1.0\textwidth]{msgcons.png}
\end{columns}%[t]
\end{frame}

\begin{frame}{进程(线程)调度}
当系统中有多个进程或线程处于就绪态时,操作系统需要从中选择一个放到CPU上运行。
这就是进程调度问题。实现该任务的部件称作调度器。
\end{frame}

\begin{frame}{进程的典型行为: CPU密集型与IO密集型进程}
\includegraphics[width=1.0\textwidth]{behav.png}

思考:两种进程举例?

二者关键区别:不是I/O时间长度,而是CPU时间长度
\end{frame}

\begin{frame}{进程的典型行为: CPU密集型与IO密集型进程}
CPU速度增长快于I/O速度增长,因此随着技术发展,进程倾向于越来越I/O密集。
后果:I/O密集型进程的调度显得越来越关键。
\end{frame}

\begin{frame}{进程的典型行为: CPU密集型与IO密集型进程}
基本想法:如果某I/O进程处于就绪态,则应该努力优先让其运行。为什么?(这里有个深刻原因)

思考2:如何动态识别某进程是CPU密集型还是I/O密集型?
\end{frame}

\begin{frame}{进程的典型行为: CPU密集型与IO密集型进程}
\includegraphics[width=1.0\textwidth]{util.png}
\end{frame}

\begin{frame}{进程调度的时机:何时调度?}
\begin{enumerate}
\item 新进程创建时,是继续运行父进程,还是运行新创建的子进程?
\item 当前进程退出时,CPU空闲,此时需从就绪态进程集合中选择一个运行
\item 当前进程阻塞时(I/O或者信号量引起)
\item 当发生I/O中断时,由此I/O信号导致阻塞的进程进入就绪态
\end{enumerate}
\end{frame}

\begin{frame}{抢占式调度与非抢占式调度}
\begin{itemize}
\item 时钟硬件中断信号的频率大约为50~60Hz
\item 在1个或者K个时钟中断信号处,强迫终止当前运行的进程。这类调度称为抢占式调度
\item 非抢占式调度:进程一旦运行,则除非它阻塞或者自愿放弃CPU,不剥夺其CPU使用权。
\item 抢占式调度用于分时系统;需要\alert{时钟硬件}的支持
\end{itemize}
\end{frame}

\begin{frame}{时间片轮转调度算法}
\begin{itemize}
\item 最简单、最古老、最公平、广泛使用
\item 时间片长度设置:
\begin{itemize}
\item 时间片不能太短,否则进程切换开销比例太大
\item 时间片太长,则导致交互式使用时等待时间过久
\end{itemize}
\end{itemize}
\includegraphics[width=1.0\textwidth]{round.png}
\end{frame}

\begin{frame}{基于优先级的调度算法}
\begin{itemize}
\item 时间片轮转算法的假设:所有进程都同等重要
\item 有时,有些进程比较重要(校长、院长、主任、教授、秘书、清洁工、学生)
\item 优先级调度算法:从就绪进程集合中选择优先级最高的进程运行
\item 关键数据结构:\alert{优先队列}
\item 为避免优先级高的进程独霸CPU,可以动态调整优先级
\item 为照顾I/O密集型进程,优先级可以设置为1/f, 其中f为进程在上一时间片中实际占用CPU的时间比例
\end{itemize}
\end{frame}

\begin{frame}{优先级调度与时间片轮转相结合的调度算法}
\includegraphics[width=1.0\textwidth]{comb.png}
\end{frame}

\begin{frame}{线程调度}
\includegraphics[width=1.0\textwidth]{thread.png}
\end{frame}
\end{CJK*}
\end{document}
